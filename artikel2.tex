\documentclass{dtk2}
\usepackage{graphicx,mwe}
\author{Ellen Bogen}
\address{Ellen}{Bogen}%
        {Hinter dem Berg~17\\
         54321~Neustadt\\
         \url{Ellen.Bogen@dante.de}}
\keywords{LaTeX,DTK,Test}%

\begin{document}
\title{Ein zweiter Artikel zum "`Test"'}

\maketitle

Siehe~\cite{lamport:handbuch}

Etwas Text
\begin{lstlisting}[language=C]
#include <stdio.h>
#define N 10
/* Block
 * comment */

int main()
{
    int i;

    // Line comment.
    puts("Hello world!");

    for (i = 0; i < N; i++)
    {
        puts("LaTeX is also great for programmers!");
    }

    return 0;
}
\end{lstlisting}
\begin{lstlisting}[language={[LaTeX]TeX},style=nonumber,emph={Voegel}]
    \documentclass[fontsize=11pt,paper=a4,pagesize]{scrartcl}
    \usepackage[utf8]{inputenc}% Kommentar
    \usepackage[T1]{fontenc}
    \begin{document}
      Alle Voegel sind schon da.
    \end{document}
    \end{lstlisting}
\begin{verbatim}
    \documentclass[fontsize=11pt,paper=a4,pagesize]{scrartcl}
    \usepackage[utf8]{inputenc}% Kommentar
    \usepackage[T1]{fontenc}
    \begin{document}
      Alle Voegel sind schon da.
    \end{document}
    \end{lstlisting}
\end{verbatim}

\begin{figure} \centering
  \includegraphics[width=.6\textwidth]{example-image}
  \caption{Ein Testbild, ein Testbild, ein Testbild, ein Testbild, ein
  Testbild, ein Testbild, ein Testbild, ein Testbild, ein Testbild, ein
  Testbild, ein Testbild, ein Testbild}
\end{figure}

\bibliography{\jobname}

\end{document}
