\UseRawInputEncoding
\RequirePackage{amsmath}
\documentclass{dtk}
\usepackage{graphicx,mwe}

\Author{Ellen}{Bogen}{%
%  Hinter dem Berg~17\\
%  54321~Neustadt\\
%  \Email{Ellen.Bogen@dante.de}
}
\Author*{Rainer}{Unsinn}{%
  Dorfstr.~2\\
  12345~Kleinkleckersdorf\\
  \Email{Rainer.Unsinn@dante.de}
}

\keywords{LaTeX,DTK,Test}%

\license{XX-CC-BY-NC-SA}

\addbibresource{artikel2a.bib}
\addbibresource[bibencoding=latin1]{artikel2b.bib}

\begin{document}
\title{Ein zweiter Artikel zum "`Test"' (Zweitautor: Rainer Unsinn)}

\maketitle

Siehe~\cite{lamport:handbuch} 
und~\cite{niepraschk.voss:texlivelist}

Etwas Text

\texttt{mtxrun --script fonts --list --all --pattern=Plex}

\begin{lstlisting}[language={[5.3]Lua}]
-- This example uses recursive factorial definition.
function factorial(n)
  if (n == 0) then
    return 1
  else
    return n * factorial(n - 1)
  end
end

for n = 0, 16 do
  io.write(n, "! = ", factorial(n), "\n")
end
\end{lstlisting}

Etwas Text

\begin{lstlisting}[language={[LaTeX]TeX},style=nonumber,emph={Voegel}]
    \documentclass[fontsize=11pt,paper=a4,pagesize]{scrartcl}
    \begin{document}
      Alle Voegel sind schon da.
    \end{document}
\end{lstlisting}

Etwas Text

\begin{verbatim}
    \documentclass[fontsize=11pt,paper=a4,pagesize]{scrartcl}
\end{verbatim}

Etwas Text

\begin{figure} \centering
  \includegraphics[width=.6\textwidth]{example-image}
  \caption{Ein Testbild, ein Testbild, ein Testbild, ein Testbild, ein
  Testbild, ein Testbild, ein Testbild, ein Testbild, ein Testbild, ein
  Testbild, ein Testbild, ein Testbild}
\end{figure}

\printbibliography

\end{document}
