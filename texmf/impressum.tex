\begingroup
\small

»Die \TeX{}nische Komödie« ist die Mitgliedszeitschrift von
\dante{} Der Bezugspreis ist im Mitgliedsbeitrag enthalten.
Namentlich gekennzeichnete Beiträge geben die Meinung der
Autoren wieder.  Reproduktion oder Nutzung der erschienenen
Beiträge durch konventionelle, elektronische oder beliebige andere
Verfahren ist nicht gestattet. Alle Rechte zur weiteren Verwendung
außerhalb von \dante{} liegen bei den jeweiligen Autoren.
%Verwendungen in größerem Umfang bitte zur Information bei \dante{}
%melden.

Beiträge sollten in Standard-\LaTeX-Quellcode unter Verwendung der
Dokumentenklasse \texttt{dtk} erstellt und per \mbox{E-Mail} oder
Datenträger (CD/DVD) an untenstehende Adresse
der Redaktion geschickt werden.  Sind
spezielle Makros, \LaTeX-Pakete oder Schriften notwendig, so
müssen auch diese komplett mitgeliefert werden.  Außerdem müssen sie auf
Anfrage Interessierten zugänglich gemacht werden. Weitere Informationen
für Autoren
findet man auf der Projektseite \url{http://projekte.dante.de/DTK/AutorInfo}
von \dante

\smallskip

Diese Ausgabe wurde mit \texttt{\,\InfoTeX} erstellt.
Als Stan"-dard-Schriften kamen \DTKschriftenListe\ zum Einsatz.

\smallskip
\vfill
\noindent
\begin{tabular}{@{}l@{ }l@{}}
  Erscheinungsweise: & vierteljährlich\\
  Erscheinungsort:   & Heidelberg\\
  Auf\/lage:         & 2500\\
  Herausgeber: & \Dante\\
               & Postfach 10\,18\,40\\
               & 69008 Heidelberg\\[3pt]
               & \begin{tabular}[b]{@{}ll@{}}
%                Tel.: & 0\,62\,21/2\,97\,66\\
%                Fax:  & 0\,62\,21/16\,79\,06\\
                   E-Mail: & \texttt{dante@dante.de} (\dante)\\
                           & \texttt{dtkred@dante.de} (Redaktion)
                 \end{tabular}\\[4pt]
  Druck:       & Konrad Triltsch Print und digitale Medien GmbH\\
               & Johannes-Gutenberg-Str. 1--3,
                 97199 Ochsenfurt-Hohestadt\\[4pt]
  Redaktion:    &  Herbert Vo\ss\ (verantwortlicher Redakteur)\\
  Mitarbeit:    & \MitarbeiterListe
\end{tabular}

\vfill

\parbox{\textwidth}{% Warum ist hier Box nötig?
Redaktionsschluss für Heft
\ifcase\DTKissue                   \or
  2/\DTKyear: 15.\,April \DTKyear  \or
  3/\DTKyear: 15.\,Juli \DTKyear   \or
  4/\DTKyear: 15.\,Oktober \DTKyear\or
  1/\the\numexpr\DTKyear+1\relax: 15.\,Januar \the\numexpr\DTKyear+1\relax
  % TODO: Als Makro auslagern? \DTKissueTOmonth
\fi
\hfill \mbox{ISSN \DTKissn}}

\endgroup
\endinput


