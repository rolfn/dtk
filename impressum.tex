\begingroup
\small

»Die \TeX{}nische Komödie« ist die Mitgliedszeitschrift von
\dante{} Der Bezugspreis ist im Mitgliedsbeitrag enthalten.
Namentlich gekennzeichnete Beiträge geben die Meinung der
Autoren wieder.  Reproduktion oder Nutzung der erschienenen
Beiträge durch konventionelle, elektronische oder beliebige andere
Verfahren ist nicht gestattet. Alle Rechte zur weiteren Verwendung
außerhalb von \dante{} liegen bei den jeweiligen Autoren.
%Verwendungen in größerem Umfang bitte zur Information bei \dante{}
%melden.

Beiträge sollten in Standard-\LaTeX-Quellcode unter Verwendung der
Dokumentenklasse \texttt{dtk} erstellt und per \mbox{E-Mail} oder
Datenträger (CD/DVD) an untenstehende Adresse
der Redaktion geschickt werden.  Sind
spezielle Makros, \LaTeX-Pakete oder Schriften notwendig, so
müssen auch diese komplett mitgeliefert werden.  Außerdem müssen sie auf
Anfrage Interessierten zugänglich gemacht werden. Weitere Informationen
für Autoren
findet man auf der Projektseite \url{http://projekte.dante.de/DTK/AutorInfo}
von \dante

\smallskip

%This is LuaTeX, Version beta-0.70.2-2012052410 (TeX Live 2012) (format=lualatex 2012.7.8)  9 AUG 2012 15:59
%This is LuaTeX, Version beta-0.79.3 (TeX Live 2015/dev) (rev 5140)  (format=lualatex 2015.2.17)  24 FEB 2015 16:52
%This is LuaTeX, Version beta-0.80.0 (TeX Live 2015) (rev 5238)



Diese Ausgabe wurde mit
\ifluatex
\texttt{LuaTeX, Version beta-0.80.0 (TeX Live 2015) (rev 5238)}
\else
\texttt{pdfTeX 3.1415926-2.3-1.40.12 (\TeX Live 2012)}
\fi
erstellt.
Als Stan"-dard-Schriften kamen Linux Libertine, Linux Biolinum%
%\TeX\ Gyre Pagella, \TeX\ Gyre Heros%
\ifluatex, \else\ und \fi DejaVu Mono
\ifluatex und XITS %\TeX\ Gyre Pagella
Math \fi
zum Einsatz.

\smallskip
\vfill
\noindent
\begin{tabular}{@{}l@{ }l@{}}
  Erscheinungsweise: & vierteljährlich\\
  Erscheinungsort:   & Heidelberg\\
  Auf\/lage:         & 2500\\
  Herausgeber: & \Dante\\
               & Postfach 10\,18\,40\\
               & 69008 Heidelberg\\[3pt]
               & \begin{tabular}[b]{@{}ll@{}}
%                Tel.: & 0\,62\,21/2\,97\,66\\
%                Fax:  & 0\,62\,21/16\,79\,06\\
                   E-Mail: & \texttt{dante@dante.de} (\dante)\\
                           & \texttt{dtkred@dante.de} (Redaktion)
                 \end{tabular}\\[4pt]
  Druck:       & Konrad Triltsch Print und digitale Medien GmbH\\
               & Johannes-Gutenberg-Str. 1--3,
                 97199 Ochsenfurt-Hohestadt\\[4pt]
  Redaktion:    &  Herbert Vo\ss\ (verantwortlicher Redakteur)\\
  Mitarbeit:    & %\raisebox{-\height}{\begin{minipage}{.75\textwidth}
%                  \begin{multicols}3 \raggedright
%                   \makebox[0.25\linewidth][l]{Andreas Eder}%
%                   \makebox[0.25\linewidth][l]{Rudolf Herrmann}%\\
%                   \makebox[0.25\linewidth][l]{Bertram Hoffmann}	%\\
%                   \makebox[0.25\linewidth][l]{Klaus H"oppner}		%	\\
                   \makebox[0.25\linewidth][l]{Gert Ingold}%
                     \makebox[0.25\linewidth][l]{Eberhard Lisse}%
%                  \makebox[0.25\linewidth][l]{Manfred Lotz}
%                     \makebox[0.25\linewidth][l]{Jürgen Lübeck}\\
                  \makebox[0.25\linewidth][l]{Rolf Niepraschk} \\%
&                   \makebox[0.25\linewidth][l]{Heiko Oberdiek}%
%                   \makebox[0.25\linewidth][l]{Günter Partosch}%
                   \makebox[0.25\linewidth][l]{Christine Römer}\\%   TODO: Makro daraus machen?
%                  \makebox[0.25\linewidth][l]{Volker RW Schaa}%
%		\makebox[0.25\linewidth][l]{Gert Seidl}
%               \makebox[0.25\linewidth][l]{Martin Sievers}%\\%
%		\makebox[0.25\linewidth][l]{Dominik Waßenhoven}%
%                   \makebox[0.25\linewidth][l]{Uwe Ziegenhagen}
%                 \end{multicols}
               %\end{minipage}}

\end{tabular}

\vfill

\parbox{\textwidth}{% Warum ist hier Box nötig?
Redaktionsschluss für Heft
\ifcase\DTKissue                   \or
  2/\DTKyear: 15.\,April \DTKyear  \or
  3/\DTKyear: 15.\,Juli \DTKyear   \or
  4/\DTKyear: 15.\,Oktober \DTKyear\or
  1/\the\numexpr\DTKyear+1\relax: 15.\,Januar \the\numexpr\DTKyear+1\relax
  % TODO: Als Makro auslagern?
\fi
\hfill \mbox{ISSN \DTKissn}}

\endgroup
\endinput


