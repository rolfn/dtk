\documentclass[%
,full
,color
,Ausgabe=4
%,korr
%,Jahr=2016
%,Monat=11
]{dtk2}

\begin{document}

\maketitle

Beispiel einer Liste (4*enumerate)

\begin{enumerate}
  \item Erster Listenpunkt, Stufe 1
  \begin{enumerate}
    \item Erster Listenpunkt, Stufe 2
    \begin{enumerate}
      \item Erster Listenpunkt, Stufe 3
      \begin{enumerate}
        \item Erster Listenpunkt, Stufe 4
        \item Zweiter Listenpunkt, Stufe 4
      \end{enumerate}
      \item Zweiter Listenpunkt, Stufe 3
    \end{enumerate}
    \item Zweiter Listenpunkt, Stufe 2
  \end{enumerate}
  \item Zweiter Listenpunkt, Stufe 1
\end{enumerate}

Beispiel einer Liste (4*itemize)

\begin{itemize}
  \item Erster Listenpunkt, Stufe 1
  \begin{itemize}
    \item Erster Listenpunkt, Stufe 2
    \begin{itemize}
      \item Erster Listenpunkt, Stufe 3
      \begin{itemize}
        \item Erster Listenpunkt, Stufe 4
        \item Zweiter Listenpunkt, Stufe 4
      \end{itemize}
      \item Zweiter Listenpunkt, Stufe 3
    \end{itemize}
    \item Zweiter Listenpunkt, Stufe 2
  \end{itemize}
  \item Zweiter Listenpunkt, Stufe 1
\end{itemize}

Beispiel einer Liste (4*description)

\begin{description}
  \item[Erster] Listenpunkt, Stufe 1
  \begin{description}
    \item[Erster] Listenpunkt, Stufe 2
    \begin{description}
      \item[Erster] Listenpunkt, Stufe 3
      \begin{description}
        \item[Erster] Listenpunkt, Stufe 4
        \item[Zweiter] Listenpunkt, Stufe 4
      \end{description}
      \item[Zweiter] Listenpunkt, Stufe 3
    \end{description}
    \item[Zweiter] Listenpunkt, Stufe 2
  \end{description}
  \item[Zweiter] Listenpunkt, Stufe 1
\end{description}

\end{document}



